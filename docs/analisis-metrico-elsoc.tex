% Options for packages loaded elsewhere
\PassOptionsToPackage{unicode}{hyperref}
\PassOptionsToPackage{hyphens}{url}
\PassOptionsToPackage{dvipsnames,svgnames,x11names}{xcolor}
%
\documentclass[
  12pt,
]{book}
\usepackage{amsmath,amssymb}
\usepackage{setspace}
\usepackage{iftex}
\ifPDFTeX
  \usepackage[T1]{fontenc}
  \usepackage[utf8]{inputenc}
  \usepackage{textcomp} % provide euro and other symbols
\else % if luatex or xetex
  \usepackage{unicode-math} % this also loads fontspec
  \defaultfontfeatures{Scale=MatchLowercase}
  \defaultfontfeatures[\rmfamily]{Ligatures=TeX,Scale=1}
\fi
\usepackage{lmodern}
\ifPDFTeX\else
  % xetex/luatex font selection
\fi
% Use upquote if available, for straight quotes in verbatim environments
\IfFileExists{upquote.sty}{\usepackage{upquote}}{}
\IfFileExists{microtype.sty}{% use microtype if available
  \usepackage[]{microtype}
  \UseMicrotypeSet[protrusion]{basicmath} % disable protrusion for tt fonts
}{}
\makeatletter
\@ifundefined{KOMAClassName}{% if non-KOMA class
  \IfFileExists{parskip.sty}{%
    \usepackage{parskip}
  }{% else
    \setlength{\parindent}{0pt}
    \setlength{\parskip}{6pt plus 2pt minus 1pt}}
}{% if KOMA class
  \KOMAoptions{parskip=half}}
\makeatother
\usepackage{xcolor}
\usepackage[left=4cm, right=3cm, top=2.5cm, bottom=2.5cm]{geometry}
\usepackage{longtable,booktabs,array}
\usepackage{calc} % for calculating minipage widths
% Correct order of tables after \paragraph or \subparagraph
\usepackage{etoolbox}
\makeatletter
\patchcmd\longtable{\par}{\if@noskipsec\mbox{}\fi\par}{}{}
\makeatother
% Allow footnotes in longtable head/foot
\IfFileExists{footnotehyper.sty}{\usepackage{footnotehyper}}{\usepackage{footnote}}
\makesavenoteenv{longtable}
\usepackage{graphicx}
\makeatletter
\def\maxwidth{\ifdim\Gin@nat@width>\linewidth\linewidth\else\Gin@nat@width\fi}
\def\maxheight{\ifdim\Gin@nat@height>\textheight\textheight\else\Gin@nat@height\fi}
\makeatother
% Scale images if necessary, so that they will not overflow the page
% margins by default, and it is still possible to overwrite the defaults
% using explicit options in \includegraphics[width, height, ...]{}
\setkeys{Gin}{width=\maxwidth,height=\maxheight,keepaspectratio}
% Set default figure placement to htbp
\makeatletter
\def\fps@figure{htbp}
\makeatother
\setlength{\emergencystretch}{3em} % prevent overfull lines
\providecommand{\tightlist}{%
  \setlength{\itemsep}{0pt}\setlength{\parskip}{0pt}}
\setcounter{secnumdepth}{5}
\usepackage[utf8]{inputenc}
\usepackage[spanish,es-tabla]{babel}
\usepackage[fixlanguage]{babelbib}
\usepackage{geometry}
\geometry{letterpaper,left=2cm,top=2cm, right=2cm}
\usepackage{times}           
\usepackage{caption}
\captionsetup[figure, table]{labelfont={bf},labelformat={default},labelsep=period}
\usepackage{graphicx}
\usepackage{booktabs}
\usepackage{longtable}
\usepackage{array}
\usepackage{multirow}
\usepackage{wrapfig}
%\usepackage{float}
\usepackage{colortbl}
\usepackage{xcolor}
\usepackage{pdflscape}
\usepackage{tabu}
\usepackage{threeparttable}
\usepackage{pdfpages}
\usepackage{hyphsubst}
\usepackage{floatrow}
\floatsetup[figure]{capposition=top}
\floatsetup[table]{capposition=top}
\usepackage{booktabs}
\usepackage{longtable}
\usepackage{array}
\usepackage{multirow}
\usepackage{wrapfig}
\usepackage{float}
\usepackage{colortbl}
\usepackage{pdflscape}
\usepackage{tabu}
\usepackage{threeparttable}
\usepackage{threeparttablex}
\usepackage[normalem]{ulem}
\usepackage{makecell}
\usepackage{xcolor}
\ifLuaTeX
  \usepackage{selnolig}  % disable illegal ligatures
\fi
\IfFileExists{bookmark.sty}{\usepackage{bookmark}}{\usepackage{hyperref}}
\IfFileExists{xurl.sty}{\usepackage{xurl}}{} % add URL line breaks if available
\urlstyle{same}
\hypersetup{
  pdftitle={Informe de Análisis de Propiedades Métricas ELSOC 2016-2022},
  pdfauthor={Equipo ELSOC},
  colorlinks=true,
  linkcolor={blue},
  filecolor={Maroon},
  citecolor={Blue},
  urlcolor={Blue},
  pdfcreator={LaTeX via pandoc}}

\title{Informe de Análisis de Propiedades Métricas ELSOC 2016-2022}
\author{Equipo ELSOC}
\date{}

\begin{document}
\maketitle

{
\hypersetup{linkcolor=}
\setcounter{tocdepth}{1}
\tableofcontents
}
\listoffigures
\listoftables
\setstretch{1.5}
\hypertarget{presentaciuxf3n}{%
\chapter*{Presentación}\label{presentaciuxf3n}}
\addcontentsline{toc}{chapter}{Presentación}

El Estudio Longitudinal Social de Chile (ELSOC) es una encuesta desarrollada para analizar intertemporalmente la evolución del conflicto y cohesión en la sociedad chilena, basándose en modelos conceptuales descritos en la literatura nacional e internacional que abordan dichas materias. Se orienta a examinar los principales antecedentes, factores moderadores y mediadores, así como las principales consecuencias asociadas al desarrollo de distintas formas de conflicto y cohesión social en Chile. Su objetivo fundamental es constituirse en un insumo empírico para la comprensión de las creencias, actitudes y percepciones de los chilenos hacia las distintas dimensiones de la convivencia y el conflicto, y como éstas cambian a lo largo del tiempo.

Esta encuesta fue diseñada por investigadores pertenecientes al Centro de Estudios de Conflicto y Cohesión Social (COES). COES está patrocinado por la Universidad de Chile y la Pontificia Universidad Católica de Chile, y cuenta con la Universidad Diego Portales y la Universidad Adolfo Ibáñez como instituciones asociadas. Si desea obtener más información sobre COES, visite la página web de COES (www.coes.cl/). COES es una iniciativa que desde 2013 cuenta con el financiamiento del Fondo de Financiamiento de Centros de Investigación en Áreas Prioritarias (FONDAP) de la Comisión Nacional de Investigación Científica y Tecnológica (CONICYT)1, organismo dependiente del Ministerio de Educación de Chile. El levantamiento de datos de ELSOC se licita públicamente cada 2 años, y ha sido adjudicado en todas sus mediciones al Centro MicroDatos de la Universidad de Chile (CMD).

El presente documento tiene como objetivo llevar a cabo un exhaustivo análisis métrico de la batería de indicadores y escalas que conforman el Estudio Longitudinal Social de Chile (ELSOC) a lo largo de las distintas olas realizadas hasta fecha. Para lograr dicho objetivo empleamos herramientas estadísticas fundamentales, las que se centran específicamente en dos aspectos clave: el análisis de correlaciones y el cálculo del coeficiente Alfa de Cronbach.

\hypertarget{anuxe1lisis-de-correlaciones}{%
\section*{Análisis de correlaciones}\label{anuxe1lisis-de-correlaciones}}
\addcontentsline{toc}{section}{Análisis de correlaciones}

El análisis de correlaciones constituye una parte esencial del análisis métrico, destinada a evaluar la naturaleza y la fuerza de las relaciones lineales entre las variables seleccionadas a través de las diferentes olas del ELSOC. Este proceso permitirá identificar patrones y tendencias que contribuirán a una comprensión más profunda de la interconexión entre las distintas dimensiones de nuestro estudio.

El análisis de correlación se lleva a cabo cuando uno o dos ítems conformar un concepto específico. Por ejemplo, en el módulo ``Sociodemográfico'' el concepto especifico ``edad'' es medido con un único ítem ``Edad del entrevistado'' a lo largo de cada una de las olas del estudio, por lo tanto, tendríamos correlaciones inter-ola del ítem. Por otra parte, cuando se tiene dos ítems por concepto se realiza la correlación entre ítems al interior de cada ola e inter-ola. Lo anterior busca verificar la consistencia de los conceptos específicos medidos con dos ítems a lo largo del tiempo.

\hypertarget{cuxe1lculo-del-coeficiente-alfa-de-cronbach}{%
\section*{Cálculo del Coeficiente Alfa de Cronbach}\label{cuxe1lculo-del-coeficiente-alfa-de-cronbach}}
\addcontentsline{toc}{section}{Cálculo del Coeficiente Alfa de Cronbach}

La fiabilidad de las mediciones del ELSOC será rigurosamente evaluada mediante el cálculo del coeficiente alfa de Cronbach. Este indicador proporciona una medida de la consistencia interna de la escala de medición utilizada, asegurando la confiabilidad y validez de los ítem que conformar los conceptos específicos del estudio.

A través de este análisis métrico, buscamos no solo explorar las asociaciones entre las variables sino también garantizar la solidez y la coherencia de nuestras mediciones. Este documento proporcionará una visión detallada de las mediciones empleadas en el ELSOC.

El cálculo de los coeficientes de Alfa de Cronbach se aplicará a los conceptos específicos que estén conformados por más de dos ítems, además este proceso se llevará en dos etapas. Primero, se realiza en cálculo de los coeficientes al interior de cada ola, estos con el objetivo de verificar la consistencia interna de las escalas. Posteriormente, se realizar correlaciones inter-olas de los valores promedios de los alfas de cada concepto especifico, con el objetivo de verificar el comportamiento de las escalas entre olas.

\hypertarget{anuxe1lisis-de-correlaciones-1}{%
\chapter{Análisis de correlaciones}\label{anuxe1lisis-de-correlaciones-1}}

\hypertarget{ejemplo}{%
\section{Ejemplo}\label{ejemplo}}

En construcción.

\hypertarget{muxf3dulo-1-caracterizaciuxf3n-sociodemogruxe1fica}{%
\section{Módulo 1: Caracterización Sociodemográfica}\label{muxf3dulo-1-caracterizaciuxf3n-sociodemogruxe1fica}}

\hypertarget{muxf3dulo-2-redes-sociales-e-interacciones-inter-grupales}{%
\section{Módulo 2: Redes sociales e Interacciones inter-grupales}\label{muxf3dulo-2-redes-sociales-e-interacciones-inter-grupales}}

En construcción.

\hypertarget{muxf3dulo-3-territorio}{%
\section{Módulo 3: Territorio}\label{muxf3dulo-3-territorio}}

En construcción.

\hypertarget{muxf3dulo-4-ciudadanuxeda-y-democracia}{%
\section{Módulo 4: Ciudadanía y Democracia}\label{muxf3dulo-4-ciudadanuxeda-y-democracia}}

En construcción.

\hypertarget{muxf3dulo-5-legitimidad-y-desigualdad-social}{%
\section{Módulo 5: Legitimidad y desigualdad social}\label{muxf3dulo-5-legitimidad-y-desigualdad-social}}

En construcción.

\hypertarget{muxf3dulo-6-conflicto-social}{%
\section{Módulo 6: Conflicto social}\label{muxf3dulo-6-conflicto-social}}

En construcción.

\hypertarget{muxf3dulo-7-salud-y-bienestar}{%
\section{Módulo 7: Salud y bienestar}\label{muxf3dulo-7-salud-y-bienestar}}

En construcción.

\hypertarget{muxf3dulo-8-guxe9nero}{%
\section{Módulo 8: Género}\label{muxf3dulo-8-guxe9nero}}

En construcción.

\hypertarget{alpha-de-cronbach}{%
\chapter{Alpha de Cronbach}\label{alpha-de-cronbach}}

\hypertarget{ejemplo-1}{%
\section{Ejemplo}\label{ejemplo-1}}

En construcción.

\hypertarget{muxf3dulo-1-caracterizaciuxf3n-sociodemogruxe1fica-1}{%
\section{Módulo 1: Caracterización Sociodemográfica}\label{muxf3dulo-1-caracterizaciuxf3n-sociodemogruxe1fica-1}}

\hypertarget{muxf3dulo-2-redes-sociales-e-interacciones-inter-grupales-1}{%
\section{Módulo 2: Redes sociales e Interacciones inter-grupales}\label{muxf3dulo-2-redes-sociales-e-interacciones-inter-grupales-1}}

En construcción.

\hypertarget{muxf3dulo-3-territorio-1}{%
\section{Módulo 3: Territorio}\label{muxf3dulo-3-territorio-1}}

En construcción.

\hypertarget{muxf3dulo-4-ciudadanuxeda-y-democracia-1}{%
\section{Módulo 4: Ciudadanía y Democracia}\label{muxf3dulo-4-ciudadanuxeda-y-democracia-1}}

En construcción.

\hypertarget{muxf3dulo-5-legitimidad-y-desigualdad-social-1}{%
\section{Módulo 5: Legitimidad y desigualdad social}\label{muxf3dulo-5-legitimidad-y-desigualdad-social-1}}

En construcción.

\hypertarget{muxf3dulo-6-conflicto-social-1}{%
\section{Módulo 6: Conflicto social}\label{muxf3dulo-6-conflicto-social-1}}

En construcción.

\hypertarget{muxf3dulo-7-salud-y-bienestar-1}{%
\section{Módulo 7: Salud y bienestar}\label{muxf3dulo-7-salud-y-bienestar-1}}

En construcción.

\hypertarget{muxf3dulo-8-guxe9nero-1}{%
\section{Módulo 8: Género}\label{muxf3dulo-8-guxe9nero-1}}

En construcción.

\hypertarget{dudas-y-consultas}{%
\chapter{Dudas y consultas}\label{dudas-y-consultas}}

En caso de que tenga dudas sobre aspectos de la validación de los datos no cubiertos por el presente Informe de Análisis de Propiedades Métricas, las cuales no puedan ser resueltas por otros medios, puede comunicarse con el Equipo de Encuestas COES al correo electrónico \url{elsoc@coes.cl}. El equipo procesará su solicitud y tratará de contestar en el más breve plazo.

Para más información respecto a los datos de ELSOC, se puede revisar:

\begin{itemize}
\tightlist
\item
  \href{https://manual-metodologico-elsoc.netlify.app/}{Manual Metodológico}
\item
  Informe \href{https://radiografia-cambio-social-2016-2022.netlify.app/}{Radriografía del Cambio Social en Chile 2016-2022}
\end{itemize}

\hypertarget{anexos}{%
\chapter*{Anexos}\label{anexos}}
\addcontentsline{toc}{chapter}{Anexos}

\end{document}
